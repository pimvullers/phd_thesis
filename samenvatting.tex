\begin{otherlanguage}{dutch}

\chapter*{Samenvatting}
\addcontentsline{toc}{chapter}{Samenvatting}

We leven in een wereld waarin computers een steeds grotere rol spelen. Daarom 
wordt het alsmaar belangrijker om gebruikers en objecten digitaal te kunnen
identificeren. Om dit te bereiken gebruiken veel bestaande systemen unieke 
nummers, denk bijvoorbeeld aan je burgerservicenummer (BSN). Dit is een 
eenvoudige oplossing, maar dit maakt het ook gemakkelijk om iemands handelingen
te volgen. Een privacyvriendelijker alternatief is om attributen te gebruiken 
voor authenticatie en autorisatie in de digitale wereld. Deze attributen, of 
eigenschappen, kunnen gecombineerd worden in een soort cryptografische container
die dan gecertificeerd wordt door een bevoegde autoriteit. Met behulp van deze 
attributen kan een gebruiker dan geauthentiseerd worden, om toegang te krijgen 
tot bepaalde systemen of om gebruik te maken van een dienst, door enkel de 
relevante gebruikerseigenschappen te delen.

In dit proefschrift bespreken we drie privacyvriendelijke technologie\"{e}n die
authenticatie op basis van attributen mogelijk maken. Deze technologie\"{e}n, 
waarvoor we effici\"{e}nte implementaties voor op een chipkaart hebben 
ontwikkeld, zijn:

\paragraph{Self-blindable Credentials}
Deze technologie maakt gebruik van cryptografie op basis van elliptische
krommen. Het systeem laat de meeste berekeningen door de kaartlezer uitvoeren, 
wat een zeer compacte implementatie op de chipkaart mogelijk maakt. Helaas is 
de ondersteuning voor elliptische kromme cryptografie op chipkaarten beperkt 
tot de standaard algoritmen. Dit maakt de ontwikkeling van varianten op 
deze technologie lastig. In vergelijking met de andere technologie\"{e}n bieden
de Self-blindable Credentials slechts beperkte mogelijkheden.

\paragraph{U-Prove}
De uitgifte- en verificatieprotocollen van U-Prove zijn gebaseerd op de methode
voor geblindeerde handtekeningen van Schnorr en zijn zero-knowledge proofs 
(bewijzen waarbij geen kennis overgedragen wordt). Deze technologie biedt op dit
moment de snelste implementatie voor verificatie met attributen. Met
betrekking tot privacy is er echter een groot probleem: U-Prove beschermt niet 
tegen het aan elkaar koppelen van meerdere verificatiesessies. Dit betekent dat de verificatiegegevens functioneren als een pseudoniem voor de gebruiker.

\paragraph{Identity Mixer}
Deze technologie is gebaseerd op het werk van Camenisch en Lysyanskaya dat een 
methode voor digitale handtekeningen definieert. Deze methode bevat protocollen
voor geblindeerde handtekeningen en zero-knowledge proofs, die gebruikt worden 
voor de uitgifte en verificatie van de attributen. In vergelijking met de andere
implementaties behalen we hier niet de beste prestaties, maar deze technologie 
biedt wel uitgebreide mogelijkheden en geeft een goede bescherming. Zo kunnen de
attributen bijvoorbeeld meerdere keren gebruikt worden zonder dat de 
verschillende verificatiesessies traceerbaar worden.

\paragraph{}
Het doel van het onderzoek in dit proefschrift was om \emph{effici\"{e}nte 
chipkaart implementaties te ontwikkelen voor op attributen gebaseerde 
authenticatie} en \emph{het vergelijken van de verschillende technologie\"{e}n}.
Dit heeft geresulteerd in een gedetailleerde beschrijving en bespreking van de 
bovengenoemde technologie\"{e}n en de bijbehorende chipkaart implementaties\footnote{De broncode van de chipkaart implementaties is beschikbaar op \url{https://github.com/pimvullers/}.}. Deze implementaties bieden, ten tijden van het schrijven van dit proefschrift, de beste prestaties voor 
attributen op een chipkaart.

Daarnaast heeft de succesvolle ontwikkeling van deze implementaties de basis
gelegd voor het IRMA project. Dit is een lopend onderzoeks- en 
ontwikkelingsproject dat zich richt op authenticatie op basis van attributen en 
de toepassing daarvan in de praktijk. Voor meer informatie over het IRMA 
project, zie \url{https://www.irmacard.org/}.

\end{otherlanguage}

\cleardoublepage
