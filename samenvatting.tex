\chapter*{Samenvatting}
\addcontentsline{toc}{chapter}{Samenvatting}

We leven in een wereld waarin computers een steeds grotere rol spelen. Daarom 
wordt het alsmaar belangrijker om gebruikersen objecten digitaal te kunnen
identificeren. Om dit te bereiken gebruiken veel bestaande systemen unieke 
nummers, denk bijvoorbeeld aan je burgerservicenummer (BSN). Dit is een 
eenvoudige oplossing, maar dit maakt het ook gemakkelijk om iemands handelingen
te volgen. Een privacy-vriendelijker alternatief is om verificatiegegevens op 
basis van attributen te gebruiken voor authenticatie en autorisatie in de 
digitale wereld. Zulke verificatiegegevens zijn een soort cryptografische 
container voor attributen, of eigenschappen, die gecertificeerd zijn door een 
bevoegde autoriteit. Met behulp van deze attributen kan een gebruiker 
geauthentiseerd worden om toegang te krijgen tot bepaalde systemen of gebruik
te maken van een dienst door alleen de relevante gebruikerseigenschappen te 
delen.

In dit proefschrift bespreken we drie technologie\"{e}n van op attributen 
gebaseerde verificatiegegevens, waarvoor we effici\"{e}nte smartcard
implementaties ontwikkeld hebben. Deze technologie\"{e}n zijn:

\paragraph{Self-blindable Credentials}
Deze technologie maakt gebruik van cryptografie op basis van elliptische
krommen. Het systeem laat de meeste berekeningen door de terminal uitvoeren, 
wat een zeer compacte implementatie op de smartcard mogelijk maakt. Helaas is 
de ondersteuning voor elliptische kromme cryptografie op smartcards beperkt tot
de standaard algoritmen. Dit maakt de ontwikkeling van andere varianten van 
deze technologie lastig. De self-blindable credentials hebben slechts beperkte 
mogelijkheden in vergelijking met de andere technologie\"{e}n.

\paragraph{U-Prove}
De uitgifte- en verificatieprotocollen van U-Prove zijn, respectievelijk,
gebaseerd op de geblindeerde handtekening methode van Schnorr en Schnorr's 
zero-knowledge proofs. Deze technologie biedt de snelste implementatie voor 
verificatie met attributen. Met betrekking tot privacy is er echter een groot 
probleem: U-Prove beschermt niet tegen het aan elkaar koppelen van meerdere 
verificatiesessies. Dit betekent dat deze verificatiegegevens functioneren als
een pseudoniem voor de gebruiker.

\paragraph{Identity Mixer}
Deze technologie is gebaseerd op de digitale handtekening methode van 
Camenisch-Lysyanskaya, welke een blind signature protocol bevat, dat gebruikt kan worden voor de 
uitgifte van de verificatiegegevens, en zero-knowledge proofs voor de 
verificatie van attributen. De prestaties van deze implementatie zijn niet de 
beste vergeleken met de andere, maar deze technologie biedt uitgebreide 
mogelijkheden en heeft een goede bescherming. Verificatiegegevens kunnen 
hierbij meerdere keren gebruikt worden zonder dat het traceerbaar wordt.

\paragraph{}
Het doel van het onderzoek was om efficiënte smartcard implementaties te maken 
voor verificatiegegevens op basis van attributen en

