\chapter*{Samenvatting}
\addcontentsline{toc}{chapter}{Samenvatting}

We leven in een wereld waarin computers een steeds grotere rol spelen. Daarom 
wordt het alsmaar belangrijker om digitale entiteiten te identificeren. Om dit
te bereiken gebruiken veel bestaande systemen unieke herkenningspunten. Dit is
een eenvoudige oplossing, maar dit maakt het ook gemakkelijk om iemands 
handelingen te volgen. Een privacy-vriendelijk alternatief is om 
verificatiegegevens op basis van attributen te gebruiken voor authenticatie en
autorisatie. Zulke verificatiegegevens zijn een soort cryptografische container
voor attributen, ook wel gebruikerseigenschappen, welke gecertificeerd zijn 
door een autoriteit. Met behulp van deze attributen kan een gebruiker 
geauthentiseerd worden om toegang te krijgen tot bepaalde informatie of gebruik
te maken van een dienst door alleen de relevante gebruikerseigenschappen te 
delen.

In dit proefschrift bespreken we drie technologieën betreffende 
verificatiegegevens gebaseerd op attributen, waarvoor we efficiënte smartcard
implementaties hebben ontwikkeld. Deze technologieën zijn:

\paragraph{Self-blindable Credentials}
Deze verificatiegegevens zijn gebaseerd op elliptische kromme cryptografie met 
bilineaire paren. Deze technologie laat de terminal de berekeningen doen, wat 
een zeer compacte implementatie op de smartcard mogelijk maakt. Helaas is de 
ondersteuning voor elliptische kromme cryptografie beperkt tot de standaard 
algoritmes. Dit maakt de ontwikkeling van andere varianten op deze technologie
moeilijk. In vergelijking met de andere technologieën heeft de self-blindable 
credentials beperkte mogelijkheden.

\paragraph{U-Prove}
De uitgifte en verificatieprotocollen zijn respectievelijk gebaseerd op de 
blind signature scheme van Schnorr en de zero-knowledge proofs. Deze 
technologie biedt de snelste implementatie voor verificatie met attributen. 
Met betrekking tot privacy is er een groot probleem: U-Prove beschermt niet 
tegen het aan elkaar koppelen van meerdere verificatiesessies. Dit betekent
dat deze verificatiegegevens werken als een pseudoniem van de gebruiker.

\paragraph{Identity Mixer}
Deze technologie is gebaseerd op de signature scheme van Camenisch-Lysyanskaya,
welke een blind signature protocol bevat, dat gebruikt kan worden voor de 
uitgifte van de verificatiegegevens, en zero-knowledge proofs voor de 
verificatie van attributen. De prestaties van deze implementatie zijn niet de 
beste vergeleken met de andere, maar deze technologie biedt uitgebreide 
mogelijkheden en heeft een goede bescherming. Verificatiegegevens kunnen 
hierbij meerdere keren gebruikt worden zonder dat het traceerbaar wordt.

\paragraph{}
Het doel van het onderzoek was om efficiënte smartcard implementaties te maken 
voor verificatiegegevens op basis van attributen en

