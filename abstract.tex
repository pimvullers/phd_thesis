\chapter*{Abstract}
\addcontentsline{toc}{chapter}{Abstract}

In a world where computers are involved in most aspects of our lives, it
becomes more and more important to digitally identify entities. To achieve 
this goal, many existing systems use unique identifiers. This is a simple
solution, but also makes it easy to trace the user's actions. A 
privacy-friendly alternative is to use attribute-based credentials as a 
basis for authentication and authorisation. Such credentials serve as a 
cryptographic container for attributes, that is, properties of the user,
which are certified by an authority. With these attributes the user can 
be authenticated to access a resource or receive a service solely on the 
properties that are relevant for that specific resource or service.

In this thesis we discuss three attribute-based credential technologies 
for which we have developed efficient smart card implementations. These 
technologies are:

\paragraph{Self-blindable Credentials}
These credentials are based on elliptic curve cryptography with bilinear 
pairings. This technology shifts the computational burden to the terminal 
which makes a very compact smart card implementation possible. 
Unfortunately the support for elliptic curve cryptography on smart cards 
is limited to standard algorithms, which made it hard to develop other 
variants of this technology. This results in a minimal feature set 
compared to the other technologies.

\paragraph{U-Prove}
The U-Prove issuance and verification protocols are, respectively, based 
on Schnorr's blind signature scheme and zero-knowledge proofs. This 
technology offers the fastest implementation for attribute verification. 
With respect to privacy there is only one important drawback: U-Prove 
does not protect against linking multiple verification sessions to each 
other. This means that these credentials basically act as a pseudonym for 
the user.

\paragraph{Identity Mixer}
This technology is based on the Camenisch-Lysyanskaya signature scheme 
which provides a blind signature protocol, which can be used for 
credential issuance, and zero-knowledge proofs for attribute verification. 
The performance of this implementation is not the best among these 
technologies, but this technology provides a broad feature set and offers 
proper unlinkability. This makes it possible to use a credential 
multiple times without becoming traceable.

\paragraph{}
The goal of the research presented in this thesis has been to \emph{develop 
efficient smart card implementations of attribute-based credentials} and 
\emph{compare various cryptographic systems for attribute-based credentials}. 
This has resulted in a detailed description and discussion of the
technologies listed above and the smart card implementations\footnote{The
source code of these implementations is available at 
\url{http://github.com/pimvullers/}.} for each of these technologies, which are
the most efficient implementations at the time of writing.

Furthermore, the successful development of these implementations laid
the foundation for the IRMA project. This is an on-going research and
development project focusing on attribute-based credentials and their
use in practice. For more information concerning the IRMA project,
please visit \url{https://www.irmacard.org/}.
