\chapter*{Abstract}
\addcontentsline{toc}{chapter}{Abstract}

In a world where computers are involved in most aspects of our lives, it
becomes more and more important to digitally identify entities. Many
existing systems use unique identifiers to achieve this goal, which is a
simple solution, but also easily allows for privacy infringements. A
more privacy-preserving alternative is to use attribute-based
credentials as a basis for authentication and authorization. Such
credentials serve as a (certified) cryptographic container for
attributes, that is, properties of the user. With these attributes the
user can be authenticated  solely on the properties that are relevant to
access a resource or receive a service.

In this thesis we describe the three attribute-based credential
technologies for which we have developed efficient smart card
implementations. These technologies are:
\begin{itemize}
  \item Self-blindable credentials, by Verheul, which are based on elliptic
   curve cryptography with bilinear pairings. In this technology the
   computational burden is shifted to the terminal which makes a very
   compact smart card implementation possible. Unfortunately the
   elliptic curve support on smart cards is limited to standard
   algorithms which made it hard to develop other variants of this
   technology, which results in a few features compared to the other
   technologies.

  \item U-Prove, by Brands and Microsoft, is based on Schnorr's blind
   signature scheme for credential issuance and zero-knowledge proofs
   for attribute verification. This technology offers the fastest
   implementation, but has one important drawback: it does not provide
   multi-show unlinkability, which means that multiple attribute
   verifications using the same credential can be linked to each other.

  \item Identity Mixer, by IBM, is based on the Camenisch-Lysyanskaya
   signature scheme which provides a blind signature protocol for
   credential issuance and zero-knowledge proofs for attribuet
   verification. While it's performance is less that U-Prove, due to
   the cryptographic primitives involved, it does offer multi-show
   unlinkability which makes it possible to use a credential multiple
   times without becoming traceable.
\end{itemize}

The goal of the research presented in this thesis has been to \emph{develop 
efficient smart card implementations of attribute-based credentials} and 
\emph{compare various cryptographic systems for attribute-based credentials}. 
This has resulted in a detailed description and discussion of the
technologies listed above and the efficient smart card implementations
for each of these technologies.

Furthermore, the successful development of these implementations laid
the foundation for the IRMA project. This is an on-going research and
development project focusing on attribute-based credentials and their
use in practice. For more information concerning the IRMA project,
please visit \url{https://www.irmacard.org/}.
