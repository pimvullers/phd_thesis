\chapter{Discussion\label{chp:discussion}}

In the previous chapters we described the attribute-based credential
technologies and the performance of their smart card implementations
individually. In this chapter we compare them with each other to determine their
(relative) strong and weak points.

\section{Attribute-based Credential Technologies}

When we compare the technologies on the cryptographic level it is clear that
different approaches can be taken to achieve the same goal. During issuance\index{credential issuance} both
U-Prove\index{U-Prove} and Identity Mixer\index{Identity Mixer} use a blind signature protocol\index{blind signature} to construct a
credential, whereas the self-blindable credentials\index{self-blindable credentials} use a regular signature
scheme. This can be explained by the fact that in the latter case the user's
public key is used for issuing the credential, whereas the former involves the
user's secret. Another difference at this stage is that Identity Mixer\index{Identity Mixer} supports
committed values to be include in the credential while U-Prove and the
self-blindable credentials require all the attribute values to be known to the
issuer\footnote{The U-Prove technology does provide an information field that
is hidden from the issuer, but this field is always disclosed during
verification which makes it different form a regular attribute.}.

Issuer unlinkability\index{unlinkability!issuer} is satisfied by each of the technologies. Identity Mixer\index{Identity Mixer}
and the self-blindable credentials\index{self-blindable credentials} require the issuer's signature\index{issuer's signature} in the
credential to be randomised to achieve this, while the U-Prove issuance protocol
results in an unlinkable signature. The signature randomisation\index{signature randomisation} also provides
(native) multi-show unlinkability\index{unlinkability!multi-show}, whereas the U-Prove\index{U-Prove} technology requires
multiple signatures in order to provide this privacy property.

When it comes to credential verification\index{credential verification} the self-blindable credentials\index{self-blindable credentials} provide
the most basic scheme. Selective disclosure\index{selective disclosure} of attributes is not applicable
since a credential only contains a single attribute statement\index{attribute!statement}. Also, because the
attribute statement is contained in a regular certificate\index{certificate} it cannot be used to
prove properties of the attribute value, such as proving that the current date
is within the validity period specified in the attribute. Both U-Prove\index{U-Prove} and Identity Mixer\index{Identity Mixer} offer
selective disclosure of attributes and support the construction of
zero-knowledge proofs\index{zero-knowledge proof} in which properties of the attributes can be proved.

Another aspect is the type of cryptography used by the different technologies.
Because the self-blindable credentials\index{self-blindable credentials} are based on elliptic curve cryptography\index{elliptic curve cryptography}
they are rather compact and require less communication between card and
terminal. U-Prove\index{U-Prove} can also be implemented using elliptic curve cryptography,
since it relies on the discrete logarithm\index{discrete logarithm problem} problem. While this would have been
an ideal solution to eliminate the communication overhead, we do not have any
cards that provide a proper interface which would allow us to implement U-Prove
based on elliptic curve cryptography. Still, the prime-order subgroup
construction used by U-Prove provides an advantage over the RSA-based approach
used by Identity Mixer\index{Identity Mixer}: it allows for modular reduction of the exponents in the
protocols.

\begin{savenotes}
\begin{table}[t]
  \centering
  \caption{Comparison of attribute-based credential technologies.}
  \label{tbl:tech-comparison}
  \renewcommand{\tabcolsep}{1.25mm}
  \renewcommand{\arraystretch}{1.25}

  \begin{tabular}{l|c|c|c|}
      & \,Self-blindable\, & \multirow{2}{*}{~~~~U-Prove~~~~} & \multirow{2}{*}{Identity Mixer} \\
      & Credentials & & \\\hline\hline
    issuer unlinkability     & \checkmark & \checkmark & \checkmark \\\hline
    multi-show unlinkability & \checkmark & indirectly\footnote{Multi-show unlinkability
      for U-Prove can be realised by issuing multiple credentials for the same
      set of attributes which can later be verified independently.} &
      \checkmark \\\hline\hline
    blind signatures         &             & \checkmark & \checkmark \\\hline
    committed values         &             &             & \checkmark \\\hline
    signature randomisation  & \checkmark &             & \checkmark \\\hline
    selective disclosure     &             & \checkmark & \checkmark \\\hline
    zero-knowledge proofs    &             & \checkmark & \checkmark \\\hline\hline
    elliptic curve cryptography & \checkmark & not used & not supported\footnote{%
      Camenisch and Lysyanskaya~\cite{CamenischLysyanskaya04} also
      describe an elliptic curve based signature scheme which can serve as a
      basis for attribute-based credentials, but this is not used in Identity
      Mixer.}\\\hline
\end{tabular}
\end{table}
\end{savenotes}

The above mentioned differences and similarities are summarised in
Table~\ref{tbl:tech-comparison}. Note that this overview only focuses on the
aspects of the technologies that we looked into. Other interesting overviews are
provided by Lapon et al.~\cite{LaponKDN2011,Lapon2012}, focusing on revocation
strategies, and Corella~\cite{Corella2011a,Corella2011b}, focusing on the use
of attribute-based credentials in the context of the United States national
strategy for trusted identities in cyberspace.

\section{Attribute-based Credentials on Smart Cards}

While we are not the first to implement attribute-based credentials on smart
cards, we do provide, to the best of our knowledge, the most efficient
implementations (see Section~\ref{sec:perf-comparison}) with the most
functionality. When we consider the existing implementations we can distinguish
a few different approaches concerning the use of smart cards.

First, a smart card can be used as a means of \emph{hardware-protection}\index{hardware-protection} for a
credential. In this scenario the card performs only \emph{a fraction} of the
issuance and verification protocols. This is motivated by the constrained\index{constrained resources}
resources of smart cards. Brands~\cite[Chapter 6]{Brands2000} proposes to use
this method for smart card integration in the U-Prove\index{U-Prove}
technology~\cite{U-Prove_Overview2011} (see Section~\ref{sec:UP-smartcard}),
whereas Bichsel~\cite{Bichsel2007} uses a similar construction to implement
protection for Identity Mixer\index{Identity Mixer!credential} credentials.

In strong contrast to the minimalist hardware-protection\index{hardware-protection} implementation, Tews
and Jacobs~\cite{TewsJacobs09} developed an implementation of attribute-based
credentials that performs \emph{all} operations on a smart card. A comparison
between these approaches is given in Table~\ref{tbl:approaches}. This
\emph{credentials on a smart card} approach does requires more resources on the
card, but it also solves the main disadvantage of the hardware-protection
approach: the smart card cannot be used independently, since it is tied to
computational (and storage) resources external to the card. This means that it
requires a specific, card matching terminal, like the user's PC, to run
the protocols.

\begin{table}
  \centering
  \caption{Comparison between the \emph{device-protection of credentials}
    approach and the \emph{credentials on a smart card} approach.}
  \label{tbl:approaches}
  \renewcommand{\tabcolsep}{1.25mm}
  \renewcommand{\arraystretch}{1.25}
  \begin{tabular}{l|p{47mm}|p{47mm}|}
     & device-protection of credentials & credentials on a smart card \\\hline
    characteristics &
      add-on security measure &
      full protocol implementation \\\hline
    card stores &
      only the device-protection attribute or secret &
      all attributes, other credential values \\\hline
    card computes &
      short zero-knowledge proof for the device-protection attribute &
      complete issuance and verification protocols \\\hline
    advantages &
      fast, lightweight, protect any number credentials
        using a single card pre-issued devices &
      independent use of the card, no need to trust the terminal \\\hline
    disadvantages &
      trusted terminal required &
      requires more card resources \\\hline
  \end{tabular}
\end{table}

The remaining implementations are based on \emph{anonymous authentication}\index{authentication!anonymous} of
the smart card after which it is trusted to provide valid attribute statements\index{attribute!statement}.
To this end, Balasch~\cite{Balasch2008} and Sterckx~\cite{Sterckx09} have
implemented direct anonymous attestation\index{direct anonymous attestation} (see Section~\ref{sec:DAA}) and
Bichsel et~al.~\cite{BichselCGS2009} have chosen to implement Identity Mixer\index{Identity Mixer}
without any attributes. These solutions use the unlinkability properties of the
Camenisch-Lysyanskaya signature\index{Camenisch-Lysyanskaya signature} scheme to achieve the anonymous authentication\index{authentication!anonymous}
of the card. This approach is also used in the German identity card (nPA)~\cite{Kugler2010}, except that
a different authentication\index{authentication} scheme is used (see Section~\ref{sec:nPA}).

\section{Smart Card Performance\label{sec:perf-comparison}}

When comparing the smart card implementations we have to take into account that
they have been developed on different platforms, and more importantly on
different hardware. To be precise, the self-blindable credentials\index{self-blindable credentials!applet} applet runs on
an NXP SmartMX J3A081 chip with the Java Card\index{Java Card} platform (JCOP v2.4.1 R3), the
U-Prove\index{U-Prove!application} application runs on an Infineon SLE66 chip with the MULTOS\index{MULTOS} platform
(I4F(1-1-2) on 360PE(M)) and the Identity Mixer\index{Identity Mixer!application} application runs on the Infineon
SLE78 chip with the MULTOS\index{MULTOS} platform (ML3-36K-R1).

\begin{table}[t]
  \centering
  \caption{Performance comparison between the NXP SmartMX chip and the Infineon
    SLE66 and SLE78 chips (time in milliseconds for 100 successive operations).}
  \label{tab:comparison}
  \renewcommand{\tabcolsep}{1.25mm}
  \renewcommand{\arraystretch}{1.25}
  \begin{tabular}{l|c|c|c|c|c|c|}
     & \multicolumn{2}{c|}{SmartMX} & \multicolumn{2}{c|}{SLE66} & \multicolumn{2}{c|}{SLE78} \\
     & contact & wireless & contact & wireless & contact & wireless \\\hline
    SHA-1 RAM        & 1110 & 1136 &  5120 &  5274 &  866 &  943 \\\hline
    SHA-1 EEPROM     & 1442 & 1466 &  6125 &  6308 & 1188 & 1285 \\\hline
    RSA-1024 RAM     &  772 &  777 &  1016 &  1060 &  668 &  877 \\\hline
    RSA-1024 EEPROM  & 1941 & 1952 &  2936 &  3041 & 2449 & 2898 \\\hline
    RSA-2048 RAM     & 1926 & 1950 & 14289 & 14898 & 1055 & 1449 \\\hline
    RSA-2048 EEPROM  & 3838 & 3865 & 17237 & 17956 & 3473 & 4192 \\\hline
  \end{tabular}
\end{table}

In order to compare the raw performance of the underlying hardware we developed
a small test application that performs some basic operations\footnote{A message
digest computation using SHA-1; and two RSA\index{RSA!encryption} encryptions using a random public
key, one with a 1024 bits modulus and one with a 2048 bits modulus.} and stores
the outcome either in RAM\index{RAM} or EEPROM.\index{EEPROM} The results of these tests are summarised
in Table~\ref{tab:comparison}. From this we can conclude that the SLE66 is
overall slower than the other cards, with a huge performance penalty when
computing an RSA operation (which is basically a modular exponentiation) with a
modulus of 2048 bits\footnote{Probably the cryptographic co-processor of the
SLE66 chip does not support this operation directly such that a software
solution is needed to handle operations involving a 2048 bits modulus.}. The
timings of the SLE78 and SmartMX chips are much closer to each other. Here we
notice that the SLE78 is in principle faster than the SmartMX, but also that
there is a larger difference between the contact and wireless (or contactless)
interfaces as well as between the memory used (RAM or EEPROM).

\subsection{Credential Issuance}

With this in mind we first look at the performance of the implementations during
issuance.\index{credential!issuance} The issuance process for the self-blindable credentials\index{self-blindable credentials} does not
involve any computations on the card, since it only has to store the credential.
This makes it the most efficient implementation at this point since both U-Prove\index{U-Prove}
and Identity Mixer\index{Identity Mixer} have to perform a blind signature\index{blind signature} protocol to issue
credentials. To make the comparison between those two easier we put their
issuance performance graphs (Figure~\ref{fig:issue} and Figure~\ref{fig:issuance})
next to each other in Figure~\ref{fig:comparison-issuance}.

\begin{figure}[ht]
  \centering
  \begin{subfigure}[b]{0.45\textwidth}
  \includegraphics[scale=.95]{images/uprove-issuance.mps}
    \caption{U-Prove}
    \label{fig:uprove-issuance}
  \end{subfigure}
  \begin{subfigure}[b]{0.45\textwidth}
  \includegraphics[scale=.95]{images/idemix-issuance.mps}
    \caption{Identity Mixer}
    \label{fig:idemix-issuance}
  \end{subfigure}

  \caption[Credential issuance performance for various numbers of attributes.]{
    Credential issuance performance for various numbers of attributes
    (\raisebox{-.8\dp\strutbox}{\includegraphics{images/box-dark.mps}}:~computation time,
      \raisebox{-.8\dp\strutbox}{\includegraphics{images/box-light.mps}}:~overhead).}
  \label{fig:comparison-issuance}
\end{figure}

Based on these graphs it is clear that the Identity Mixer\index{Identity Mixer!implementation} implementation offers
a clear improvement over the U-Prove\index{U-Prove!issuance} issuance times. Not only are the absolute
values better, the extra time it takes to issue more attributes does not
increase as much as with the U-Prove\index{U-Prove!implementation} implementation. However, we also need to
take the hardware platform into account, which in this case is in favour of the
Identity Mixer implementation. Furthermore, our U-Prove implementation is built
according to the first revision~\cite{U-Prove_Crypto2011} of version 1.1 of the
U-Prove cryptographic specification, whereas later revisions~\cite{U-Prove_Crypto2013},
include an optimised issuance protocol in which the computation of the signature
value $z'$ is mostly performed by the issuer (as described in Section~\ref{sec:UP-issuance}). Taking
this optimisation and a switch to the SLE78 platform into account, the U-Prove
implementation should get a reasonable improvement in its issuance performance
bringing it closer to the results we got from the Identity Mixer implementation.

\subsection{Selective Disclosure of Attributes}

Since a self-blindable credential only contains a single attribute, there is no
selective disclosure option in the credential verification protocol. In this
case the user just chooses which attribute(s) to reveal, and hence which
credential(s) to use. Thus, to show multiple attributes the protocol has to be
performed multiple times, which also means that the transaction time is
multiplied. This in strong contrast to the other technologies, where revealing
more attributes, from the same credential, actually reduces the running time, as
can be seen in Figure~\ref{fig:2attr-comparison}.

\begin{figure}
  \centering
  \begin{subfigure}[b]{0.33\textwidth}
    \includegraphics[scale=.95]{images/sbcred-2attr-compare.mps}
    \caption{Self-blindable Credentials}
    \label{fig:sbcred-2attr}
  \end{subfigure}
  \begin{subfigure}[b]{0.32\textwidth}
    \includegraphics[scale=.95]{images/uprove-2attr-compare.mps}
    \caption{U-Prove}
    \label{fig:uprove-2attr}
  \end{subfigure}
  \begin{subfigure}[b]{0.32\textwidth}
    \includegraphics[scale=.95]{images/idemix-2attr-compare.mps}
    \caption{Identity Mixer}
    \label{fig:idemix-2attr}
  \end{subfigure}

  \caption[Credential verification performance with two attributes.]{
    Credential verification performance with two attributes
    (\raisebox{-.8\dp\strutbox}{\includegraphics{images/box-dark.mps}}:~computation time,
      \raisebox{-.8\dp\strutbox}{\includegraphics{images/box-light.mps}}:~overhead).}
  \label{fig:2attr-comparison}
\end{figure}

\begin{figure}
  \centering
  \begin{subfigure}[b]{0.48\textwidth}
    \includegraphics[scale=.95]{images/uprove-5attr-compare.mps}
    \caption{U-Prove}
    \label{fig:uprove-5attr}
  \end{subfigure}
  \begin{subfigure}[b]{0.48\textwidth}
    \includegraphics[scale=.95]{images/idemix-5attr-compare.mps}
    \caption{Identity Mixer}
    \label{fig:idemix-5attr}
  \end{subfigure}

  \caption[Credential verification performance with five attributes.]{
    Credential verification performance with five attributes
    (\raisebox{-.8\dp\strutbox}{\includegraphics{images/box-dark.mps}}:~computation time,
      \raisebox{-.8\dp\strutbox}{\includegraphics{images/box-light.mps}}:~overhead).}
  \label{fig:5attr-comparison}
\end{figure}

Comparing the results from Figures~\ref{fig:2attr-comparison}
and~\ref{fig:5attr-comparison}, it is clear that the U-Prove technology offers
the best verification performance. The computation time for the verification of
a U-Prove credential is overall lower than the computation times of the other
technologies, even when a U-Prove credential contains five attributes of which
only a single attribute is revealed. This performance could even be improved
when the SLE78 chip can be used instead of the SLE66. Hence we can conclude that
the signature randomisation\index{signature randomisation} performed (on the card) by Identity Mixer\index{Identity Mixer} and the self-blindable
credentials has a significant impact on the running time of the verification
protocol.

While the multi-show unlinkability\index{unlinkability!multi-show} property has a negative effect on the running
time for Identity Mixer\index{Identity Mixer} and the self-blindable credentials\index{self-blindable credentials}, it requires less
storage on the card then U-Prove\index{U-Prove}. This is due to the fact that to achieve
multi-show unlinkability the card has to store multiple U-Prove tokens. Given
that storage space is rather limited on smart cards\footnote{A typical modern smart
card only has 36 to 144 KB of EEPROM available for storing application data.},
this is a serious drawback of the U-Prove technology.

\clearpage

\section{Final Remarks}

The goal of the research presented in this thesis has been to
\begin{center}\it
  develop efficient smart card implementations of attribute-based credentials
\end{center}
and
\begin{center}\it
  compare various cryptographic systems for attribute-based credentials.
\end{center}

Compared to the existing implementations mentioned in the previous section, we
have made a clear performance improvement with all of our implementations. These
implementations are not only faster, but also provide \emph{full} credentials
on a smart card instead of the partial solutions that have been developed to
cope with the smart card shortcomings. With transaction times around, or even
below one second we can conclude that these are, to the best of our knowledge,
the first implementations that offer an acceptable performance for practical
use.

The development of these prototypes allowed us to analyse the different
technologies both from a technical and functional perspective. Due to the
maturity and the multi-show unlinkability feature of the Identity Mixer
\index{Identity Mixer} technology that smart card implementation has been
selected for use in a pilot project called \emph{I Reveal My Attributes} or
IRMA \index{IRMA project} for short. This pilot aims to gain more experience in
the practical use of these kinds of privacy-preserving technologies and the
usability of smart card implementations therein. Please visit
\url{https://www.irmacard.org/} for more information and the latest news
on this project.

%The IRMA smart card is an excellent example of our innovations in the area of
%privacy-friendly identity management. It demonstrates what the attribute-based
%credential technologies are capable of and that they are a viable alternative
%for smart card-based solutions.

This IRMA project is a direct consequence of the demonstrated efficiency of our
smart card implementations, in particular for Identity Mixer. This shows the
impact and innovative power of our work in privacy-friendly identity management.
The IRMA project has the wider goal of demonstrating the broad applicability of
attribute-based credential technologies, and of the availability of a viable
alternative for current smart card-based solutions.
The availability of these privacy-friendly alternatives and the growing interest
for privacy-by-design, in particular in privacy regulations, should lead to
further innovations that can replace the traditional privacy-unfriendly identity
solutions (typically based on unique identifiers).