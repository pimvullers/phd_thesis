\chapter{Preliminaries}

\section{Elliptic Curve Cryptography}

\subsection{Elliptic Curves}

\begin{definition}[Elliptic Curve]\label{def:elliptic_curve}
  Given a finite field $\F{q}$ containing $q$ elements, where $q$ is a prime
  power, an elliptic curve $E$ over $\F{q}$ is defined by the equation
  \begin{equation}\label{eqn:elliptic_curve}
    E: y^2 = x^3 + a x + b
  \end{equation}
  where $a, b \in \F{q}$ are the curve parameters and $4 a^3 + 27 b^2 \neq 0$.
  The set of all points $P = (x, y)$ on this curve $E$ is denoted as
  \begin{equation*}
    E(\F{q}) = \{ (x, y) \in \F{q} \times \F{q} : y^2 = x^3 + a x + b \}
                         \cup \{ \infty \}
  \end{equation*}
  where $\infty$ is a special point called the \emph{point at infinity}.
\end{definition}

\subsubsection{Operations}

\begin{enumerate}
  \item \emph{Point Negation}
    If $P = (x, y) \in E(\F{q})$ then $(x, y) + (x, -y) = \infty$. The point
    $(x, -y)$ is denoted by $-P$ and is called the \emph{negative of $P$}.
    Also, $-\infty = \infty$.
  \item \emph{Point addition}
    Let $P = (x_P, y_P) \in E(\F{q})$ and $Q = (x_Q, y_Q) \in E(\F{q})$, where
    $P \neq \pm Q$. Then $P + Q = R = (x_R, y_R) \in E(\F{q})$, where
    \begin{equation*}
      x_R = \left(\dfrac{y_Q - y_P}{x_Q - x_P}\right)^2 - x_P - x_Q
      \text{ and }
      y_R = \left(\dfrac{y_Q - y_P}{x_Q - x_P}\right) (x_P - x_R) - y_P
      \text{.}
    \end{equation*}
    Also, $P + \infty = \infty + P = P$ for all $P \in E(\F{q})$.
  \item \emph{Point doubling}
    Let $P = (x_P, y_P) \in E(\F{q})$, where $P \neq -P$. Then
    $2P = R = (x_R, y_R) \in E(\F{q})$, where
    \begin{equation*}
      x_R = \left(\dfrac{3 x_P^3 + a}{2 y_P}\right)^2 - 2 x_P
      \text{ and }
      y_R = \left(\dfrac{3 x_P^3 + a}{2 y_P}\right) (x_P - x_R) - y_P
      \text{.}
    \end{equation*}
    Also, $2\infty = \infty$.
  \item \emph{Point multiplication}
    Given an integer $k$ and a point $P$ the multiplication is denoted as
    $k \cdot P$. To compute this multiplication various methods exist. A basic
    solution is the repeated-double-and-add method given in Algorithm~\ref{alg:ec_point_multiplication}.
\end{enumerate}

\begin{algorithm}
  \caption{Elliptic curve point negation: $R = -P$}
  \label{alg:ec_point_addition}

  \begin{algorithmic}[1]
    \Function{ECPointAddition}{$P$, $Q$}
      \If{$P = \infty$}
        \Return $\infty$
      \Else
        \State $(x_P, y_P) \gets P$
      \EndIf

      \smallskip
      \Return $(x_P, -y_P)$
    \EndFunction
  \end{algorithmic}
\end{algorithm}


\begin{algorithm}
  \caption{Elliptic curve point addition: $R = P + Q$}
  \label{alg:ec_point_addition}

  \begin{algorithmic}[1]
    \Function{ECPointAddition}{$P$, $Q$}
      \If{$P = \infty$}
        \Return $Q$
      \ElsIf{$P = Q$}
        \Return $2 \cdot P$
      \ElsIf{$P = -Q$ or $Q = \infty$}
        \Return $\infty$
      \Else
        \State $(x_P, y_P) \gets P$
        \State $(x_Q, y_Q) \gets Q$
      \EndIf
      \medskip

      \State $x \gets \left(\dfrac{y_Q - y_P}{x_Q - x_P}\right)^2 - x_P - x_Q$ % FIXME: \mod q
      \State $y \gets \left(\dfrac{y_Q - y_P}{x_Q - x_P}\right) (x_P - x_R) - y_P$ % FIXME: \mod q

      \smallskip
      \Return $(x, y)$
    \EndFunction
  \end{algorithmic}
\end{algorithm}

\begin{algorithm}
  \caption{Elliptic curve point doubling: $R = 2 \cdot P$}
  \label{alg:ec_point_doubling}

  \begin{algorithmic}[1]
    \Function{ECPointDoubling}{$P$}
      \If{$P = \infty$}
        \Return $P$
      \Else
        \State $(x_P, y_P) \gets P$
      \EndIf
      \medskip

      \State $x_R \gets \left(\dfrac{3 x_P^3 + a}{2 y_P}\right)^2 - 2 x_P$ % FIXME: \mod q
      \State $y_R \gets \left(\dfrac{3 x_P^3 + a}{2 y_P}\right) (x_P - x_R) - y_P$ % FIXME: \mod q

      \smallskip
      \Return $(x_R, y_R)$
    \EndFunction
  \end{algorithmic}
\end{algorithm}

\begin{algorithm}
  \caption{Elliptic curve point multiplication: $R = k \cdot P$ (repeated-double-and-add; right-to-left binary method)}
  \label{alg:ec_point_multiplication}

  \begin{algorithmic}[1]
    \Function{ECPointMultiplication}{$k$, $P$}
      \State $(k_{l-1}, \dots, k_1, k_0) \gets k$ \Comment{binary representation of $k$}
      \State $R \gets \infty$

      \For{$i$ from $0$ to $l - 1$}
        \If{$k_i = 1$}
          \State $R \gets R + P$
        \EndIf
        \State $P \gets 2P$
      \EndFor

      \smallskip
      \Return $R$
    \EndFunction
  \end{algorithmic}
\end{algorithm}

%%[BHJMV]
For elliptic curve cryptography the field $\F{q}$ is of the form $\F{p^n}$ for
some prime number $p$ and $n \in \N$ ($p \neq 2,3)$. The solutions of
equation~(\ref{eqn:elliptic_curve}) form an abelian group with point addition
as the group operation and the point at infinity as the zero-element. The
condition $4a^3 + 27b^2 \neq 0$ is required for $E$ to be non-singular, as
required for cryptographic applications. For cryptography we need a finite
cyclic group in which the group operation is efficiently computable, but the
discrete logarithm problem is very difficult to solve. Elliptic curve groups
meet these criteria when the underlying field is finite and the size of $p$ is
at least 160~bits.

\subsubsection{Elliptic Curve Discrete Logarithm Problem}

Let $E$ be an elliptic curve over $\F{q}$ and let $P \in E$ be a point of order
$k$. Let $Q \in \langle P \rangle$ be a point generated by $P$, that is,
$Q = \alpha \cdot P$ for $\alpha$ where $0 \leq \alpha < k$. The problem of
finding the logarithm $\alpha$ for given $P$ and $Q$ is called the elliptic
curve discrete logarithm problem (ECDLP).


\subsection{Pairings}\label{sec:pairings}

A bilinear pairing is a map $\mathbb{G}_1 \times \mathbb{G}_2 \rightarrow
\mathbb{G}_T$ where $\mathbb{G}_1$ and $\mathbb{G}_2$ are typically additive
groups and $\mathbb G_T$ is a multiplicative group and the map is bilinear, that
is, linear in both components. Many pairings are used in cryptography such as the
Tate pairing, ate pairing and the most recent R-ate
pairing~\cite{Ver09}. For all these pairings one often uses specific cyclic
subgroups of $E(\mathbb{F}_{p^k})$ as $\mathbb{G}_1$ and $\mathbb{G}_2$
and $\mathbb{F}_{p^k}^*$ as $\mathbb{G}_T$.

The bilinearity property can be written as follows:
\begin{equation*}
  \begin{array}{rcl}
    e(P + P',\; Q) & = & e(P,\; Q)\cdot e(P',\; Q) \\
     & \text{and} & \\
    e(P,\; Q + Q') & = & e(P,\; Q)\cdot e(P,\; Q')
  \end{array}
\end{equation*}
As a result, $e(n\cdot P,\; m\cdot Q) = e(P,\; Q)^{n\,m}$. Pairings are used for many
(new) cryptographic protocols~\cite{BSS05}, such as short
signatures~\cite{BonehLS04}, three-party one-round key agreement~\cite{Joux04},
identity based encryption~\cite{BF01} and anonymous
credentials~\cite{CamenischL04}.

\subsubsection{Barreto-Naehrig Curves}

Pairing friendly elliptic curves are curves with a small embedding degree and
large prime-order subgroup. In 2005, Barreto and Naehrig (BN) discovered a new
method for constructing pairing friendly elliptic curves of prime order over a
prime field~\cite{BN06}. More precisely, BN curves are defined over $\F{p}$
where $p = p(u) = 36 u^4 + 36 u^3 + 24 u^2 + 6 u + 1$ for $u \in \Z$ such that
$p$ is prime. The order of a BN curve is a prime $n$ where
$n = n(u) = 36 u^4 + 36 u^3 + 18 u^2 + 6 u + 1$. Hence, a BN curve is
constructed by generating integers $u$ until both $p(u)$ and $n(u)$ are prime
numbers. The embedding degree of BN curves is 12 and we detail the parameters
for our case in Section~\ref{sec:params}.

Here we use pairings of the form $e: E(\F{p}) \times E(\F{p^k}) \to \F{p^k}$,
obtained by taking $\G{1} = \G{2} = \F{p^k}$ and using the obvious inclusion
$\F{p} \to \F{p^k}$ in the first argument. The number $k$ is known in this
context as the embedding degree. As previously mentioned, one uses $k = 12$ for
BN-curves.

\subsection{DH and DSA for Elliptic Curves}\label{sec:dhdsa}

The Diffie-Hellman key agreement protocol (DH) and the Digital Signature
Algorithm (DSA) are easily adapted to the ECC case as in~\cite{BSS05}
and~\cite{JM00} respectively. We recall the protocols, which in this case are
called ECDH and ECDSA.

\subsubsection{EC Diffie-Hellman Key Agreement (ECDH)}

Alice ($A$) and Bob ($B$) wish to agree on a
secret key over an insecure channel. They first agree on the set
of domain parameters $(\mathbb{F}_q, E, n,
h, G)$. Here, $E$ is an elliptic curve over
$\mathbb{F}_q$, $G$ is a generating (publicly known) point in
the elliptic curve group of order $n$ and the integer
$h$ is called the cofactor. For the cofactor we have:
$\# E(\mathbb{F}_q) = h \, n.$
Due to the security of the ECDLP one usually
selects a curve for which $h \leq 4$.
Any random point of sufficiently high order on
an elliptic curve $E$ can be used as a key.

\begin{figure}
  \centering
  \includegraphics[scale=.4]{mscs/ecdh}
  \caption{EC Diffie-Hellman key agreement protocol}\label{fig:ecdh}
\end{figure}

\paragraph{Key Agreement}

Each time a shared key is required, the following steps, as depicted in \figurename~\ref{fig:ecdh}, have to be performed.

\begin{enumerate}
  \item $A$ chooses a random secret $x$, where $1 \leq x \leq n-1$, as her private
    key and sends $B$ the corresponding public key $x \cdot G$
  \item $B$ chooses a random secret $y$, where $1 \leq y \leq n-1$, as his private
    key and sends $A$ the corresponding public key $y \cdot G$.
  \item $B$ receives $x \cdot G$ and computes the shared key as
    $k = y \cdot (x \cdot G) = (xy) \cdot G$.
  \item $A$ receives $y \cdot G$ and computes the shared key as
    $k = x \cdot (y \cdot G) = (xy) \cdot G$.
\end{enumerate}

So, they both end up with the same point as the common key: $k = xy \cdot G$.
An adversary Eve may have knowledge of $G$, $x\cdot G$,
and $y\cdot G$ but not of $x$ or $y$. She wants to determine number $xy\cdot G$. This
task is called the ``(computational) Diffie-Hellman problem for elliptic curves''.

9.2 DL/ECKAS-DH1
DL/ECKAS-DH1 is the Discrete Logarithm and Elliptic Curve Key Agreement Scheme, Diffie-Hellman
version, where each party contributes one key pair.
\textbf{Scheme options}
The following options shall be established or otherwise agreed upon between the parties to the scheme:
\begin{itemize}
  \item A secret value derivation primitive, which shall be: DLSVDP-DH, DLSVDP-DHC, ECSVDP-DH, or ECSVDP-DHC
  \item For a -DHC secret value derivation primitive, an indication as to whether or not compatibility with the corresponding -DH primitive is desired
  \item A key derivation function, which should be KDF1 (see 13.1), or a function designated for use with DL/ECKAS-DH1 in an amendment to this standard
\end{itemize}
The above information may remain the same for any number of executions of the key agreement scheme, or
it may be changed at some frequency. The information need not be kept secret.
\textbf{Key agreement operation}
A sequence of shared secret keys, K 1 , K 2 , ..., K t , shall be generated by each party by performing the
following or an equivalent sequence of steps:
\begin{enumerate}
  \item Establish the valid set of DL or EC domain parameters with which the parties’ key pairs shall be associated.
  \item Select a valid private key s for the operation, associated with the parameters established in step 1.
  \item Obtain the other party’s purported public key w′ for the operation, associated with the parameters established in step 1.
  \item (Optional) If the selected secret value derivation primitive is DLSVDP-DHC or ECSVDP-DHC, then validate that w′ is an element in the appropriate group (i.e., in GF (q) for DL or on the elliptic curve for EC; see 6.2.2 and 7.2.2); otherwise, validate that w′ is a valid public key. If any validation fails, output “invalid public key” and stop.
  \item Compute a shared secret value z from the private key s and the other party’s public key w′ with the selected secret value derivation primitive (see 9.2.1).
  \item Convert the shared secret value z to an octet string Z using FE2OSP.
  \item For each shared secret key to be agreed on (a.) Establish or otherwise agree on key derivation parameters P i for the key.
 (b.) Derive a shared secret key K i from the octet string Z and the key derivation parameters P i with the selected key derivation function (see 9.2.1).
\end{enumerate}

\subsubsection{EC Digital Signature Algorithm (ECDSA)}

The ECDSA is specified by an elliptic curve $E$ defined
over $\mathbb{F}_q$ and a publicly known point
$G \in E$ of prime order $n$. As above, a private key of Alice is a scalar $z$ and the
corresponding public key is $Q = z \cdot G \in E$. The ECDSA requires a hash
function in addition and consists of two parts as explained below.

\paragraph{Signature Generation}

In order to sign a message $m$, $A$ should perform the following steps:

\begin{enumerate}
  \item Select a random integer $k$, where $1 \leq k \leq n-1$.
  \item Compute $(x_1,\; y_1) = k \cdot G$.
  \item Compute $r = x_1\, \bmod n$. If $r = 0$, then go back to the step 1.
  \item Compute $s = k^{-1} (h(m)+z \, r)\bmod n$, where $h$ is a hash function.
    If $s = 0$, then go to step 1.
  \item $A$'s signature for the message $m$ is the pair $(r,\; s)$.
\end{enumerate}

\paragraph{Signature Verification}

In order to verify $A$'s signature on $m$, $B$ performs the following steps:

\begin{enumerate}
  \item Obtain an authenticated copy of $A$'s public key $Q$.
  \item Verify that $r$ and $s$ are integers in the interval $[1, n-1]$.
  \item Compute $w = s^{-1} \bmod n$ and $h(m)$.
  \item Compute $u_1 = h(m) \, w \bmod n$ and $u_2 = r\,w \bmod n$.
  \item Compute $(x_0,\; y_0) = u_1 \cdot G + u_2 \cdot Q$ and
    $v = x_0 \bmod n$.
  \item Accept the signature if and only if $v = r$.
\end{enumerate}
