\chapter{Self-Blindable Credentials}

\section{Chaum-Pedersen Signature Scheme}

This section presents the signature scheme~\cite{ChaumPedersen93} which will be
used as the basis for the following schemes. The public key of the scheme is
$(p, q, g, h)$, where $h \in G_q \setminus \{1\}$ and the corresponding secret key is
$x = \log_g h$. Let $m \in G_q$, be a message. The signature on $m$ consists of
$z = m^x$ plus a proof that $\log_g h = \log_m z$.

Given $m$ and $z$, consider the following protocol:
\begin{enumerate}
  \item The prover chooses $s \in \Z$, at random and computes
    $(a, b) = (g^s \mod p, m^s \mod p)$. This pair is sent to the verifier.
  \item The verifier chooses a random challenge $c \in \Z$, and sends it to the
    prover.
  \item The prover sends back $r = s + cx \mod q$
  \item The verifier accepts the proof if $g^r = ah^c$ and $m^r = bz^c$.
\end{enumerate}

\section{Boneh-Lynn-Shacham Signature Scheme}

[independent work by both Verheul and BLS]

Boneh, Lynn and Shacham~\cite{BonehLS01,BonehLS04} present a signature scheme
that works on any Gap co-Diffie-Hellman group pair $(G_1, G_2)$. The scheme
resembles the undeniable signature scheme proposed by Chaum and
Pedersen~\cite{ChaumPedersen93}. Verheul~\cite{Verheul01} even refers to this
scheme as a proofless variant of the Chaum-Pedersen scheme. Okamoto and
Pointcheval~\cite{OkamotoPointCheval01} briefly note that gap problems can give
rise to signature schemes. However, most gap problems will not lead to short
signatures. Let $(G_1, G_2)$ be $(t, \epsilon)$-Gap co-Diffie-Hellman group
pair where $|G_1| = |G_2| = p$. A signature $\sigma$ is an element of $G_1$.
The signature scheme comprises three algorithms, KeyGen, Sign, and Verify.

\begin{enumerate}
  \item Key generation. Pick random $x \in_R \Z_p$ and compute
    $v \gets g_2^x$. The public key is $v \in G_2$. The private key is $x$.
  \item Signing. Given a private key $x \in \Z_p$, and a message
    $M \in \{0, 1\}^∗$, compute $h \gets \mathcal{H}(M) \in G_1$ and
    $\sigma \gets h^x$. The signature is $\sigma \in G_1$.
  \item Verification. Given a public key $v \in G_2$, a message
    $h \gets \mathcal{H}(M) \in G_1$ and verify that $(g_2, v, h, \sigma)$ is
    a valid co-Diffie-Hellman tuple, that is $e(g_2, \sigma) = e(v, h)$. If
    so, output valid; if not, output invalid.
\end{enumerate}

Verheul~\cite{Verheul01} points out a powerful aspect about these signatures,
they are invariant under blinding. Thus, if a card chooses an arbitrary number
$b$ as blinding factor, the resulting pair
$(b\cdot P_{c}, \; b\cdot (s\cdot P_{c}))$ is again a signature, for the
private-public key pair $(b\,k_{c}, \; b\cdot P_{c} = (b\,k_{c})\cdot P)$. Each
time the card is used it can thus present a different (signed) public key, such
that the different uses cannot be linked.

\section{Verheul's Self-Blindable Certificates}

Of course, when a card presents a reader with such a pair $P_c$, $R$
the reader should not only check that $R$ is a proper signature on
$P_c$, that is, $R$ is $s\cdot P_c$, but also that the card knows the private
key corresponding to the public key $P_c$. This can be done via standard
challenge-response exchange, for example using ECDSA.

\subsection{Credential Pseudonymous Certificates}

Credential pseudonymous certificates are digital certificates that bind
credentials to users known by a pseudonym. Proof of credential possession is
given by proving possession of the private key related to the public key
referenced in the certificate.

$\{P_U, [Sig(P_U, S_N), Cert(P_N, "Pseudonym Provider statement)]\}$

Here, the public key $P_U$ of the user is signed using a private key $S_N$ of
the pseudonym provider. The corresponding public key $P_N$ of the pseudonym
provider is included in a (conventional PKI) certificate which contains the
statement corresponding to this public key.

When we consider only a single statement per public key, these certificates can
be stored in public database and can be omitted from the credential, given that
the corresponding public key can be identified.

\section{Credential Issuance}

[TODO: create issuance protocol diagram]

To obtain a credential from a credential issuer the cardholder must provide the
issuer with its public key and proof posession of the corresponding private key.
This public key can be signed by an identity provider in order to increase the
trust level of the system. When the credential issuer has verified the
authenticity of the user he will verify eligibility for the requested attribute.
Once this has been verified the issuer signs the user's public key using the
private key corresponding to the requested attribute and send this signature to
the user.

\section{Credential Verification}

\begin{figure}
  \centering
  \includegraphics[scale=.4]{mscs/textual_en}
  \caption{Protocol for proving self-blindable attributes}\label{fig:textual_en}
\end{figure}

When the user wants to utilize an attribute stored in a credential it has to
show the credential to the service provider. To prevent the service provider
from tracing her based on the public key the user first blinds her key pair and
the credential. Next it sends the results from this blinding operation to the
service provider and proofs possession of the (blinded) private key. This last
step is easily achieved by signing a challenge received from the service
provider using this private key. The service provider can then verify this
signature using the public key it received earlier.

\section{Java Card Implementation}

\subsection{Available ECC Operations}

\subsubsection{EC Diffie-Hellman}

This functionality is provided by the javacard.security.KeyAgreement class. This
class can be instantiated using the getInstance(byte algorithm, boolean externalAccess)
method which returns a KeyAgreement object that implements a certain algorithm.
The generateSecret(byte[] publicData, short publicOffset, short publicLength, byte[] secret, short secretOffset)
method can then be used to actually perform the Diffie-Hellman operation, in our
case an elliptic curve multiplication.

The standard Java Card API provides two algorithms~\cite{JavaCard-2.2.2_API_documentation}:
\begin{description}
  \item[ALG\_EC\_SVDP\_DH] Elliptic curve secret value derivation primitive, Diffie-Hellman version, as per [IEEE P1363].
  \item[ALG\_EC\_SVDP\_DHC] Elliptic curve secret value derivation primitive, Diffie-Hellman version, with cofactor multiplication, as per [IEEE P1363].
\end{description}

Unfortunately for us, the implementation of these algorithms has two drawbacks:
\begin{enumerate}
  \item According to the IEEE P1363~\cite{IEEE_P1363} standard the shared secret computation by means of ECSVDP-DH and ECSVDP-DHC only returns the $x$-coordinate of the computed point.
  \item The Java Card implementation of these algorithms computes the SHA-1 message digest of the output of the derivation primitive to yeild a 20 byte result~\cite{JavaCard-2.2.2_API_documentation}.
\end{enumerate}

Especially this last transformation of the result prevents it from being useful for any further computation, other than using it as a secret key.

\paragraph{JCOP Extension}

Luckily the JCOP platform contains some extensions to the standard Java Card API.
In particular this extension provides the KeyAgreementX.ALG\_EC\_SVDP\_DH\_PLAIN algorithm:

\begin{lstlisting}
  /**
	 * KeyAgreement algorithm ALG_EC_SVDP_DH_PLAIN is the same as ALG_EC_SVDP_DH
	 * but without SHA1 postcomputation.
	 */
\end{lstlisting}

com.nxp.id.jcopx.KeyAgreementX
KeyAgreementX.getInstance(KeyAgreementX.ALG\_EC\_SVDP\_DH\_PLAIN, false);

This removes the second drawback as mentioned before and only leaves us with the
$x$-coordinate of the result instead of a point. However, the point can be reconstructed
from this coordinate using the elliptic curve formula. By inputting the $x$ value
we can compute the $y$ value. The only unknown is the sign of the $y$-coordinate,
hence we end up with two candidate points for the multiplication result.

\subsection{Public Key and Credential Blinding}

\begin{figure}
  \centering
  \includegraphics[scale=.4]{mscs/basic}
  \caption{Protocol for proving self-blindable attributes}\label{fig:basic}
\end{figure}

\subsection{Proof of Possession of the Private Key}

\begin{figure}
  \centering
  \includegraphics[scale=.4]{mscs/optimised}
  \caption{Protocol for proving self-blindable attributes}\label{fig:optimised}
\end{figure}

\subsubsection{Using the ECDSA Signature Scheme}

This approach depends on the NatLib library developed by Hendrik
Tews~\cite{TewsJacobs09}. This library is used to blind the secret key before
it can be used in the ECDSA algorithm to sign the challenge received from the
terminal.

\subsubsection{Using the BLS Signature Scheme}

This approach exploits the KeyGeneration functionality provided by the Java Card
API. This functionality is used to generate the blinding factor and blind the
received challenge at the same time, This allows us to simply sign the blinded
challenge with the private key using the BLS signature scheme.

\section{Performance results}
