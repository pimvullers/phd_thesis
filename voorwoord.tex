\chapter*{Voorwoord}
\addcontentsline{toc}{chapter}{Voorwoord}

Het is inmiddels al weer vijf jaar geleden dat ik in Nijmegen ben gestart met 
mijn promotie traject. Aan de ene kant is het een traject dat je zelf moet 
doorlopen, maar aan de andere kant is het ook zeker een traject dat je niet 
alleen kunt doorlopen. Daarom wil ik bij deze iedereen bedanken die me tijdens 
deze periode geholpen heeft. Ik wil niemand te kort doen, maar toch wil ik een
aantal personen in het bijzonder noemen.

Om te beginnen mijn promotor, Bart, die tevens de taak van dagelijks begeleider
op zich genomen heeft. Niet alleen heb je me ondersteund bij het uitvoeren van 
mijn onderzoek en het schrijven van mijn artikelen, maar ook met het promoten 
van mijn resultaten en het opdoen van ervaring binnen het onderwijs. Ondanks je
drukke agenda was er altijd wel tijd om mijn dagelijkse beslommeringen te 
bespreken. Bart, bedankt.

Ook de artikelen waarop dit proefschrift gebaseerd is heb ik niet alleen 
geschreven. Ik wil dan ook mijn collega's, Wojciech, Gergely, Bart, Jaap-Henk
en Lejla van harte bedanken voor hun hulp. Wojciech en Erik wil ik ook graag 
bedanken voor hun kennis en ervaring met het programmeren van smartcards waar
ik veel van geleerd heb, en in dit werk veelvuldig toegepast heb.

Daarnaast wil ik ook mijn mede-promovendi bedanken voor de leuke tijd die ik 
gehad heb. Bij de promovendi school IPA was het altijd wel gezellig, in het 
bijzonder het gezelschap van Carst en Jeroen, mijn medestudenten van de TU/e,
en Wouter en Gergely, mijn collega's uit Nijmegen, en natuurlijk de organisatie
van IPA, Meivan en Tim. Ook heb ik mijn volledige promotie op een ruime kamer
gezwoegd met vele andere promovendi. Uit de gezamenlijke arbeid heb ik veel 
energie geput, maar zeker ook uit onze gesprekken en werkonderbrekingen. In het
bijzonder staan de kleine security studie projectjes samen met Gerhard en Joeri
me nog goed bij.

Tijdens mijn promotie heb ik ook de mogelijkheid gehad om bij Microsoft stage 
te lopen. Zo heb ik ervaring op kunnen doen en in de keuken kunnen kijken bij 
een onderzoeksafdeling bij een groot bedrijf. Christian, bedankt voor het
regelwerk om de stage mogelijk te maken en je begeleiding tijdens mijn tijd bij
Microsoft.

Uit mijn onderzoek is het IRMA project voortgekomen dat verder onderzoek doet
naar het gebruik van attributen in de praktijk. Er is veel werk in gaan zitten,
en ik ben blij dat ik dit niet alleen heb hoeven doen. Wouter, Gergely, Ronny,
Roland, Maarten, Jaap-Henk, Antonio, Bart en Martijn, bedankt.

Ook wil ik mijn commissieleden bedanken voor de tijd die ze gestoken hebben in 
het beoordelen van mijn proefschrift. In het bijzonder wil ik Sjouke bedanken, 
omdat je ook al mijn begeleider bent geweest tijdens mijn afstudeerproject in 
Luxemburg.

Natuurlijk heeft al dit werk ook zijn invloed gehad op mijn persoonlijke leven.
Tijdens mijn promotie zullen mijn broers Bas en Koen als mijn paranimfen 
optreden. Koen heeft ook de tijd genomen om een omslag te ontwerpen voor dit
proefschrift, en me daar veel werk mee uit handen genomen. Van mijn ouders,
Jan en Gerrie, heb ik altijd de steun gehad voor deze promotie, ook al was het
niet altijd te begrijpen waar ik dan precies mee bezig was. 

Als laatste wil ik Marlou bedanken, die het niet altijd even makkelijk heeft 
gehad tijdens mijn promotie (in het bijzonder tijdens het schrijven van dit 
proefschrift), maar me toch altijd ondersteund heeft om dit werk af te ronden. 
Bedankt.

\begin{flushright}
  \textit{Pim Vullers\\ Nooitgedacht, oktober 2014}
\end{flushright}

\cleardoublepage
